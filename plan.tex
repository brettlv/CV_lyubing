\prefix{}
\begin{rubric}{Research Plan}
\entry*[Plan 1]
The mechanism of ``changing-look'' phenomena (variation of broad emission lines and/or  line-of-sight column densities of AGN)  is still under debate. Many studies have been reported to show that the ``changing-look'' events are strongly correlated with the change of the accretion state in AGNs. We can statistically study the $L_{UV}$ and $L_\mathrm{2keV}$ correlation of CLAGNs and the evolution of optical-to-X-ray index ($\alpha_{OX}$) with luminosity to investigate the co-evolution between disk-corona system and BLR of CLAGNs using simultaneous Swift UV/X-ray data. Besides, there are 5 CLAGNs show variations of both broad emission lines and X-ray absorption column density (e.g., ESO 362-G18, NGC 2992, NGC 4151, NGC 4395, and NGC 7582). Through the spectral fitting of those CLAGNs using hard X-ray data, we can investigate the key factors that determine the ``changing-look'' such as the luminosity, accretion state, torus structure (e.g. cover factor, inclination, and light of sight column density).

\entry*[Plan 2]
The upcoming optical spectroscopic and photometric survey telescopes (e.g. LSST, CSST) would provide us rich data to search a large sample of CLAGNs and extreme variable AGNs. We can study the structure and evolution of accretion disk, BLR, and torus of those special AGNs through reverberation mapping (RM) method using long-term optical (e.g. ZTF and ASAS-SN) and IR (e.g. WISE) monitoring data. 

\entry*[Plan 3]
The birth, growth, and death of AGN activity are important for us to understand the co-evolution of SMBH and its host galaxy in the history of universe. We can search the restarted or dying AGNs and study the evolution of AGNs in different scales through multi-wavelength data (e.g. $L_\mathrm{150MHz}$, $L_\mathrm{X-ray}$, $L_\mathrm{OIII}$, and $L_\mathrm{5GHz}$) by cross-matching different surveys or catalogs (e.g. GLEAM, FIRST, WISE, and SDSS). Comparing them with normal AGNs, we can get a better comprehension about how the SMBH seeds are formed in the early phase of universe, how the growth of AGN is regulated by the episodic accretion process, and how AGN cease to be active.



\end{rubric}